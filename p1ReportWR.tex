\documentclass[12pt,a4paper]{article}

\usepackage[letterpaper]{geometry}

\usepackage{times}
\geometry{top=1.0in, bottom=1.0in, left=1.0in, right=1.0in}

\usepackage{fancyhdr}
\pagestyle{fancy}
\lhead{}
\rhead{}
\lfoot{}
\rfoot{}

\renewcommand{\headrulewidth}{0pt} 
\renewcommand{\footrulewidth}{0pt} 

\setlength\headsep{0.333in}

\usepackage{fontspec}
\setmainfont{Times New Roman}

\usepackage{graphicx}

\usepackage{hyperref}

\usepackage{caption}

\usepackage{indentfirst}

\usepackage{setspace}



\begin{document}

\begin{titlepage}
	
\vspace*{5cm}

\begin{center}

\Huge{Exercise 1} \\

\vspace{0.4cm}

\LARGE{Measurements of the Moment of Inertia}

\vspace*{1.6cm}

\Large{
\begin{tabular}{ll}
Student Name & {\fontspec{Hei}\selectfont 王韧}  Ren Wang \\
Student ID: & 516370910177 \\
Group: & 11 
\end{tabular}
}

\end{center}
\end{titlepage}

\doublespacing 
\newpage

\section{Theoretical Background}

Moment of inertia of a rigid body about an axis is a quantitative
characteristics that defines the body’s resistance (inertia) to a change of
angular velocity in rotation about that axis. 
This characteristics of the rigid body rotating about a fixed axis is determined
not only by the mass of the body, but also by its distribution. 
The moment of inertia of a rigid body about a certain rotation axis can be
calculated analytically. 
However, if the body has irregular shape or non-uniformly distributed mass, the
calculation may be di cult.
Experimental methods turn out to be more useful in such cases.

\subsection{Laws of Physics Used}

There are mainly two laws of Physics been used in the expriment. Second law of
dynamics for rotational motion can find the relationship between the rotational
acceleration and the torque at the object.

\subsubsection{Second Law of Dynamics for Rotational Motion}

\subsubsection{Parallel Axis Theorem}


\section{Apparatus}

\section{Procedure}

\section{Calculations and Results}

\section{Measurement Uncertianty Analysis}

\section{Conclusions and Discussion}

% data sheet

\end{document}