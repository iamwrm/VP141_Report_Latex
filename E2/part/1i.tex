\section{Introduction}
Fluid viscosity is one of the most important properties of fluids, determining
the fluid’s flow. 
Motion of an object in a fluid is hindered by a drag force acting in the
direction opposite to the direction of motion, i.e. opposite to the object’s
velocity. 

The magnitude of the drag force is related to the shape and speed of the object
as well as to the internal friction in the fluid.
The method used in this lab is known as Stokes’ method and it a common and
simple method for characterizing transparent or translucent fluids with high
viscosity.


% theoretical
Motion of an object in a fluid is hindered by a drag force acting in the
direction opposite to the direction of motion, i.e. opposite to the object’s
velocity. 
The magnitude of the drag force is related to the shape and speed of the object
as well as to the internal friction in the fluid.

This internal friction can be quantified by a number known as the viscosity
coefficient $\mu$.
For a spherical object with radius R moving at speed v in an infinite volume of
a liquid, the magnitude of the drag force is usually modeled as linear in the
speed
$$  F_1 = 6 \pi \mu v R  $$

When a spherical object falls vertically downwards in a fluid, it is being acted
upon by the following three forces:
The viscous force \emph{$F_1$} and the buoyancy force \emph{$F_2$} both act
upwards, and the weight of the object \emph{$F_3$} is directed downwards.
The magnitude of the buoyancy force is
$$  F_2 = \frac{4}{3} \pi R^3 \rho_1 g $$
where $\rho_1$ is the density of the fluid and $g$ is the acceleration due to
gravity. The weight of the object
$$  F_3 = \frac{4}{3} \pi R^3 \rho_2 g $$
with $\rho_2$ being the density of the object. After some time, the three forces
will balance each other
$$  F_1 + F_2 = F_3  $$
so that the net force on the object will be zero and from that instant on, the
object will be moving with constant speed $v_t$, known as the terminal speed.
Applying the condition, we can find
$$  \mu = \frac{2}{9} g R^2 \frac{\rho_2 - \rho_1 }{v_t}  $$
Therefore, the fluid viscosity can be found by measuring the terminal speed.
Taking into account that the motion with terminal speed is a motion with
constant velocity, the equation can be rewritten as
$$  \mu = \frac{2}{9} g R^2 \frac{( \rho_2 - \rho_1 ) t  }{s}  $$
where $s$ is the distance traveled in time $t$ after reaching the terminal
speed.

Since the volume of the fluid used in the measurement is not infinite, the
results are affected by some boundary effects due to the presence of the
container.

Therefore, the equation should be modified, and the formula for the
corrected magnitude of the viscous force for a infinitely long cylindrical
container with radius $R_c$ is 
$$  F_1 = 6 \pi \mu v R (1 + 2.4 \frac{R}{R_c})  $$
Consequently,
$$ \mu = \frac{2}{9} g R^2 \frac{( \rho_2 - \rho_1 ) t  }{s} (1 + 2.4
\frac{R}{R_c})  $$
Since the length $L$ of the container is limited, there may be further
corrections introduced, depending on the ratio on $\frac{R_c}{L}$. 





